%
% Portuguese-BR vertion
% 
\documentclass{report}

\usepackage{ipprocess}
% Use longtable if you want big tables to split over multiple pages.
% \usepackage{longtable}
\usepackage[utf8]{inputenc} 
\usepackage[brazil]{babel} % Uncomment for portuguese

\sloppy

\graphicspath{{./pictures/}} % Pictures dir
\makeindex
\begin{document}

\DocumentTitle{Especificação de Requisitos}
\Project{Processador Lapidopacalamba}
\Organization{Universidade Estadual de Feira de Santana}
\Version{Build 1}

\capa
\newpage
\newpage

%%%%%%%%%%%%%%%%%%%%%%%%%%%%%%%%%%%%%%%%%%%%%%%%%%
%% Revision History
%%%%%%%%%%%%%%%%%%%%%%%%%%%%%%%%%%%%%%%%%%%%%%%%%%
\chapter*{Histórico de Revisões}
  \vspace*{1cm}
  \begin{table}[ht]
    \centering
    \begin{tabular}[pos]{|m{2cm} | m{6cm} | m{3cm}|} 
      \hline
      \cellcolor[gray]{0.9}\textbf{Date} & \cellcolor[gray]{0.9}\textbf{Descrição} & \cellcolor[gray]{0.9}\textbf{Autor(s)}\\
      \hline
      21/12/2015 & Concepção e estruturação do Documento & Patricia Gomes \\ \hline      
      21/12/2015 & Finalização do Documento & Patricia Gomes \\ \hline      
      21/12/2015 & Revisão do Documento & Fábio Barros \\ \hline 
      28/01/2016 & Correções no Documento & Patricia Gomes \\ \hline
    \end{tabular}
  \end{table}

\newpage

% TOC instantiation
\tableofcontents
\newpage

%%%%%%%%%%%%%%%%%%%%%%%%%%%%%%%%%%%%%%%%%%%%%%%%%%
%% Document main content
%%%%%%%%%%%%%%%%%%%%%%%%%%%%%%%%%%%%%%%%%%%%%%%%%%
\section{Introdução}
 \subsection{Propósito do Documento}
 O propósito deste documento é descrever e especificar os requisitos que devem ser  atendidos pelo produto, de forma a satisfazer as necessidades dos solicitantes, bem como definir o produto a ser feito, para os desenvolvedores.
 
  O projeto consiste no desenvolvimento um processador capaz de executar 42 instruções. O processador desenvolvido possui 16 registradores de propósito geral, sendo que cada registrador possui a capacidade de armazenamento de 32 bits.
  
  O documento descreve a lógica de implementação de todos os requisitos do processador.

\subsection{Visão Geral do Documento}
  \begin{itemize}
   \item \textbf{Requisitos funcionais -} lista de todos os requisitos funcionais.
   \item \textbf{Requisitos não funcionais -} lista de todos os requisitos não funcionais.
  \end{itemize}

  % inicio das definições do documento
  \subsection{Definições}
    \FloatBarrier
    \begin{table}[H]
      \begin{center}
        \begin{tabular}[pos]{|m{5cm} | m{9cm}|} 
          \hline
          \cellcolor[gray]{0.9}\textbf{Termo} & \cellcolor[gray]{0.9}\textbf{Descrição} \\ \hline
          Requisitos Funcionais & Requisitos de hardware que compõem os módulos, descrevendo as ações que o mesmo deve estar apto a executar.   \\ \hline
          Requisitos Não Funcionais & Requisitos de hardware que compõem os módulos, representando as características que o mesmo deve ter, ou restrições que o mesmo deve operar. Estas características referem-se técnicas, algoritmos, tecnologias e especificidades do Sistema como um todo.  \\ \hline
        \end{tabular}
      \end{center}
    \end{table}  
  % fim

  % inicio da tabela de acronimos e abreviacoes do documento
  \subsection{Acrônimos e Abreviações}
    \FloatBarrier
    \begin{table}[H]
      \begin{center}
        \begin{tabular}[pos]{|m{2cm} | m{12cm}|} 
          \hline
          \cellcolor[gray]{0.9}\textbf{Sigla} & \cellcolor[gray]{0.9}\textbf{Descrição} \\ \hline
          FR      & Requisito Funcional  \\ \hline
          NFR     & Requisito Não Funcional  \\ \hline
          ULA     & Unidade de Lógica e Aritmética  \\ \hline
        \end{tabular}
      \end{center}
    \end{table}  
  % fim

  % inicio dos requisitos funcionais
  \section{Requisitos Funcionais}
    \subsection{Requisitos dos módulos}
    \begin{functional}
     % \requirement{name}{description}
      \requirement
      {Criação de uma Unidade de Lógica e Aritmética}
      {Para que o processador atenda ao requisito de ser capaz de executar 42 instruções, é necessário construir uma unidade responsável por executar as operações de lógica e aritmética(ULA). A ULA deve ser capaz de realizar as operações com dois operandos de 32 bits. A mesma deve possuir duas saídas, uma para apresentar o resultado que também será de 32 bits, e outra para apresentar as flags atualizadas de acordo com a operação realizada. Além disso, a ULA deve possuir um sinal de controle responsável por indicar a mesma o código da operação, ou seja, qual operação ela deverá executar no momento.}
      {Importante}
      
       \requirement
      {Criação de um Extensor de sinais}
      {Para viabilizar as operações com constantes e de saltos deve ser implementado um extensor de sinais. O mesmo deve ser capaz de extender 12 ou 16 bits para 32 bits conforme o sinal de controle enviado para o extensor. Para realizar a extensão dos sinais, o extensor deve replicar o bit de sinal.}
      {Importante}
      
       \requirement
      {Criação de um registrador de flags}
      {Deve ser criado um registrador de flags de forma a armazenar as flags atualizadas de acordo com a operação da ULA. A ULA possui uma saída de flags e para essa saída devem  ser enviadas as flags atualizadas na mesma. Um sinal de controle é enviado para permitir que a saída da ULA com as flags atualizadas sejam armazenadas no registrador de flags.}
      {Importante}
      
    \requirement
      {Atualizar Flags}
      {A medida que uma operação é realizada algumas flags deverão ser atualizadas. A atualização dessas flags se dará de acordo com a operação realizada pela ULA. As Flags a serem atualizadas são: Overflow, Sinal, Carry e Zero. A flag de sinal é atualizada se o resultado da operação for negativo. Se o resultado for igual a 0, deve-se atualizar a flag zero. No caso da flag de overflow, a mesma deve ser atualizada se houver carry sobre o bit de sinal. Por fim, a flag de carry deve ser atualizada quando houver um "vai um" após o bit de sinal.  }
      {Importante}
   
      \requirement
      {Criação de um testador de flags}
      {Para ser possível a realização de desvios condicionais deve ser implementado um módulo testador de flags que terá como função analisar uma condição e decidir se um jump condicional será ou não realizado. As condições a serem analisadas são apresentadas no documento de arquitetura. Se a condição for satisfeita e for enviando um sinal de jump false o salto não deverá ser tomado, logo para que um salto seja executado,  a condição deve ser verdadeira e o sinal de jump deve ser true, ou a condição deve ser falsa e o sinal de jump deve ser false. }
      {Importante}
      
       \requirement
      {Criação de uma memória de dados}
      {Deverá ser desenvolvida uma memória responsável por salvar/ler dados do banco de registradores de acordo com as instruções de acesso à memória (load e store). Essa memória deve possuir 2\^{}32 endereços e cada endereço armazenar 4Bytes (32bits).}
      {Importante}
      
       \requirement
      {Criação de um banco de registradores}
      {O banco de registradores deve possuir 16 registradores com capacidade de armazenamento de 32 bits. O banco deve ter como entrada os endereços dos registradores fonte, o endereço do registrador destino, e uma entrada para o dado a ser armazenado. E deve possuir como saída os dados contidos nos registradores fonte.}
      {Importante}
      
       \requirement
      {Criação de uma memória de instrução}
      {Deve existir uma memória específica para armazenar todas as instruções que deverão ser realizadas pelo processador. Essa memória deve possuir 2\^{}32 endereços e cada endereço armazenar 4Bytes (32bits)}
      {Importante}
    
      \end{functional}
      
     \subsection{Requisitos das operações}
      \begin{functional}

     \requirement
      {Operação de adição}
      {A Unidade Lógica e Aritmética recebe dois operandos inicialmente armazenados no banco de registradores e soma os dois operandos bit a bit. A operação de soma deve atualizar todas as flags.}
      {Importante}

     \requirement
      {Operação de adição com incremento}
      {A Unidade Lógica e Aritmética recebe dois operandos inicialmente armazenados no banco de registradores executa a soma dos dois operandos bit a bit e em seguida soma  ao número 1. Esta operação deve atualizar todas as flags.}
      {Importante}
      
      \requirement
      {Operação de incremento}
      {A Unidade Lógica e Aritmética recebe um operando inicialmente armazenado no banco de registradores e soma ao número 1. Esta operação deve atualizar todas as flags.}
      {Importante}
      
      \requirement
      {Operação de subtração com decremento}
      {A Unidade Lógica e Aritmética recebe dois operandos inicialmente armazenados no banco de registradores e executa a subtração fazendo uma soma  bit a bit com o segundo operando em complemento a 2 e em seguida subtrai o número 1. Essa operação deve atualizar todas as flags.}
      {Importante}
      
      \requirement
      {Operação de subtração}
      {A Unidade Lógica e Aritmética recebe dois operandos inicialmente armazenados no banco de registradores e executa a subtração fazendo uma soma  bit a bit com o segundo operando em complemento a 2. Esta operação deve atualizar todas as flags.}
      {Importante}
      
      \requirement
      {Operação de decremento}
      {A Unidade Lógica e Aritmética recebe um operando inicialmente armazenado no banco de registradores e subtrai o número 1. Esta operação deve atualizar todas as flags.}
      {Importante}
      
       \requirement
      {Operação de deslocamento lógico}
      {A Unidade Lógica e Aritmética recebe um operando inicialmente armazenado no banco de registradores e desloca seus bits à esquerda. Esta operação deve atualizar as flags de sinal, carry e zero.}
      {Importante}
      
       \requirement
      {Operação de deslocamento aritmético}
      {A Unidade Lógica e Aritmética recebe um operando inicialmente armazenado no banco de registradores e desloca seus bits à direita. Esta operação deve atualizar as flags de sinal, carry e zero.}
      {Importante}
      
       \requirement
      {Operação zeros}
      {A Unidade Lógica e Aritmética envia zero para saída. Esta operação deve atualizar apenas a flag de zero.}
      {Importante}
      
      \requirement
      {Operação and}
      {A Unidade Lógica e Aritmética recebe dois operandos inicialmente armazenados no banco de registradores e executa uma and entre os dois operandos. Esta operação deve atualizar as flags de sinal e zero.}
      {Importante}
      
      \requirement
      {Operação and com o primeiro operando negado}
      {A Unidade Lógica e Aritmética recebe dois operandos inicialmente armazenados no banco de registradores e faz o complemento a 2 do primeiro operando, em seguida executa uma and entre os dois operandos. Esta operação deve atualizar as flags de sinal e zero.}
      {Importante}
      
      \requirement
      {Operação que passa o operando B}
      {A Unidade Lógica e Aritmética recebe um operando inicialmente armazenado no banco de registradores e o envia para saída. Esta operação deve atualizar as flags de sinal e zero.}
      {Importante}
      
      \requirement
      {Operação and com o segundo operando negado}
      {A Unidade Lógica e Aritmética recebe dois operandos inicialmente armazenados no banco de registradores e faz o complemento a 2 do segundo operando, em seguida executa uma and entre os dois operandos. Esta operação deve atualizar as flags de sinal e zero.}
      {Importante}
      
      \requirement
      {Operação que passa o operando A}
      {A Unidade Lógica e Aritmética recebe um operando inicialmente armazenado no banco de registradores e o envia para saída. Esta operação deve atualizar as flags de sinal e zero.}
      {Importante}
      
      \requirement
      {Operação xor}
      {A Unidade Lógica e Aritmética recebe dois operandos inicialmente armazenados no banco de registradores e executa uma xor entre os dois operandos. Esta operação deve atualizar as flags de sinal e zero.}
      {Importante}
      
      \requirement
      {Operação or}
      {A Unidade Lógica e Aritmética recebe dois operandos inicialmente armazenados no banco de registradores e executa uma or entre os dois operandos. Esta operação deve atualizar as flags de sinal e zero.}
      {Importante}
      
      \requirement
      {Operação nand}
      {A Unidade Lógica e Aritmética recebe dois operandos inicialmente armazenados no banco de registradores, transforma os dois em negativos usando o complemento a 2 e sem seguida executa uma and entre os  operandos. Esta operação deve atualizar as flags de sinal e zero.}
      {Importante}
      
       \requirement
      {Operação xnor}
      {A Unidade Lógica e Aritmética recebe dois operandos inicialmente armazenados no banco de registradores e executa uma xor entre os dois operandos, em seguida transforma o valor obtido em negativo fazendo o complemento a 2. Esta operação deve atualizar as flags de sinal e zero.}
      {Importante}
      
      \requirement
      {Operação que passa o operando A negativo}
      {A Unidade Lógica e Aritmética recebe um operando inicialmente armazenado no banco de registradores o transforma em negativo fazendo o complemento a 2 e o envia para saída. Esta operação deve atualizar as flags de sinal e zero.}
      {Importante}
      
       \requirement
      {Operação or com o primeiro operando negado}
      {A Unidade Lógica e Aritmética recebe dois operandos inicialmente armazenados no banco de registradores, transforma o primeiro operando em negativo usando o complemento a 2 e em seguida executa uma or entre os  operandos. Esta operação deve atualizar as flags de sinal e zero.}
      {Importante}
      
       \requirement
      {Operação que passa o operando B negativo}
      {A Unidade Lógica e Aritmética recebe um operando inicialmente armazenado no banco de registradores o transforma em negativo fazendo o complemento a 2 e o envia para saída. Esta operação deve atualizar as flags de sinal e zero.}
      {Importante}
      
       \requirement
      {Operação or com o segundo operando negado}
      {A Unidade Lógica e Aritmética recebe dois operandos inicialmente armazenados no banco de registradores, transforma o segundo operando em negativo usando o complemento a 2 e sem seguida executa uma or entre os operandos. Esta operação deve atualizar as flags de sinal e zero.}
      {Importante}
      
      \requirement
      {Operação nor}
      {A Unidade Lógica e Aritmética recebe dois operandos inicialmente armazenados no banco de registradores, transforma os dois em negativos usando o complemento a 2 e sem seguida executa uma or entre os  operandos. Esta operação deve atualizar as flags de sinal e zero.}
      {Importante}
      
      \requirement
      {Operação ones}
      {A Unidade Lógica e Aritmética envia 1 para saída. Esta operação não deve atualizar nenhuma flag.}
      {Importante}
      
      \requirement
      {Operação loadlit}
      {A Unidade Lógica e Aritmética recebe o valor de uma connstante contido em um registrador do banco de registradores e o envia para saída.}
      {Importante}
      
       \requirement
      {Operação lcl}
      {Inicialmente é realizada uma operação and entre uma constante de 16 bits e o valor contido no endereço 0xffff0000 da memória, em seguida é feita uma or entre o resultado dessa and e a constante de 16 bits.}
      {Importante}
      
       \requirement
      {Operação lch}
      {Inicialmente a constante é deslocada em 16 bits e é realizada uma operação and entre a constante e o valor contido no endereço 0x0000ffff da memória, em seguida é feita uma or entre o resultado dessa and e a constante deslocada.}
      {Importante}
      
       \requirement
      {Operação load}
      {O registrador B é lido do banco de registradores e a saída da memória é escrita no banco de registradores no endereço especificado.}
      {Importante}
      
      \requirement
      {Operação store}
      {Na instrução Store os dados devem ser lidos do banco de registradores e escritos na memória, sendo que o registrador A contém o dado a ser escrito e o registrador B o endereço onde será armazenado.}
      {Importante}
      
       \requirement
      {Operação de desvio incondicional}
      {Deve realizar um salto para um endereço de instrução definido na memória de instruções, de forma a permitir a execução da instrução para onde ocorreu o salto. No desvio incondicional o valor de PC deve apontar para a instrução seguinte, esse valor da memória de instruções tem o sinal extendido e o resultado vai para a ula e da ula vai um multiplexador cuja seleção dependerá do módulo responsável por testar as flags. No jump incondicional é enviando um sinal que habilita o jump no testador de flags junto com a condição TRUE da tabela de condições (consta no documento de arquitetura). O testador de Flags envia o sinal de seleção para o mux e o mux manda para sua saída o valor que havia saído da ula e esse valor volta pra PC que indicará qual instrução deverá ser realizada.}
      {Importante}
      
       \requirement
      {Operação de desvios condicionais}
      {Deve realizar um salto para um endereço de instrução definido na memória de instruções, de forma a permitir a execução da instrução para onde ocorreu o salto. No desvio incondicional o valor de PC deve apontar para a instrução seguinte, esse valor da memória de instruções tem o sinal extendido e o resultado vai para a ula e da ula vai um multiplexador cuja seleção dependerá do módulo responsável por testar as flags. Se a condição de flags for verdadeira e o jump for true, ou se a condição for falsa e o jump for false, o multiplaxador manda para sua saída o valor que havia saído da ula e esse valor volta pra PC que insicará qual instrução deverá ser realizada.
.}
      {Importante}
      
      \requirement
      {Operação de desvios tipo and link}
      {O desvio do tipo jump and link é um desvio incondicional onde o valor de PC+1 deve ser armazenado no registrador r15 e o conteúdo do registrador RB armazenado em PC. Esse tipo de jump permite que o valor que PC tinha antes do desvio seja recuperado, logo, após o salto, PC volta a apontar para a mesma instrução que apontava antes do salto.}
      {Importante}
      
       \requirement
      {Operação de jump register}
      {Deve armazenar o conteúdo do registrador RB no PC.}
      {Importante}
      
      \requirement
      {Halt}
      {Deve realizar um salto incondicional para o endereço atual.}
      {Importante}
      
      \requirement
      {Nop}
      {Nessa instrução todos os sinais de controle devem ser zerados, de forma que nada seja registrado na memória ou no banco de registradores.}
      {Importante}
      
    \end{functional}

\section{Requisitos não Funcionais}
% Esta seção apresenta a lista de Requisitos não Funcionais do projeto.

  \begin{nonfunctional}
    \requirement
    {Armazenamento em memória de forma big-endian}
    {O armazenamento na memória deve ser feito de forma big-endian, logo, o bit mais significativo do dado deve ser armazenado na posição menos significativa da memória.}
    {Importante}

    \requirement
    {Unidade de controle hardwired}
    {A unidade de controle deve ser hardwire, logo, sinais de controle devemm ser gerados com o uso de técnicas de circuitos lógicos convencionais.}
    {Importante}
  \end{nonfunctional}


% Optional bibliography section
% To use bibliograpy, first provide the ipprocess.bib file on the root folder.
% \bibliographystyle{ieeetr}
% \bibliography{ipprocess}

\end{document}
